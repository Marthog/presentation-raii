\documentclass[xcolor=colortbl
%,draft
]{beamer}
\mode<presentation>


\usepackage{minitoc}
\usepackage{listings}
\usepackage{amsthm}
\usepackage{tikz}

% Load Theme
\usetheme[slogan=false, navigation=true, frametotal=false, myriad=false]{UzL}

\usetikzlibrary{graphs, graphdrawing, positioning}
\usegdlibrary{layered}
\usegdlibrary{force}

\title{Memory Models}
\author{Gunnar Bergmann}
\date{30. November 2015}

\begin{document}
\maketitle

\begin{frame}{Table of contents}
    \tableofcontents
\end{frame}

\section{Manual memory management}

\begin{frame}{Manual memory management}
    Problems caused by manual memory management:
    \begin{itemize}
        \item Memory leaks
        \item use after free
        \item double delete
        \item repeated resource release on all paths
        \item separation of allocation and release
        \item exceptions only with finally
    \end{itemize}
\end{frame}

\begin{frame}{Garbage Collection}
    \begin{itemize}
        \item memory overhead
        \item unpredictable collection cycle
        \item restricted to memory:
        \begin{itemize}
            \item no deterministic destruction order
            \item no immediate destruction
        \end{itemize}
    \end{itemize}
    
    $\implies$ Not always useful
\end{frame}


\begin{frame}{Local variables}
   \begin{itemize}
       \item memory automatically reclaimed after end of scope
       \item stack allocated
       \item returned by copying the value
       \item pointers become invalid after end of scope
    \end{itemize}
\end{frame}


\section{RAII}

\begin{frame}{RAII}
    \begin{itemize}
        \item Classes contain constructor and destructor
        \item Constructor allocates resource
        \item Destructor frees resource
        \item \emph{Resource Acquisition Is Initialization}
    \end{itemize}
\end{frame}


\begin{frame}[fragile, shrink]
\begin{lstlisting}[language=c++,frame=single]
class String {
private:
    char* data; // pointer to a character
public:
    // Constructor
    String(const char* s) {
        data = new char[strlen(s)+1];
        strcpy(data, s);
    }
    
    // disable copying
    String(const String&) = delete;
    
    // Destructor
    ~String() {
        delete[] data;
    }
};
\end{lstlisting}
\end{frame}

\begin{frame}[fragile]
    Add a member function for appending strings:
\begin{lstlisting}[language=c++, frame=single]
concat(const char* s) {
    char* old = data;
    int len = strlen(old)+strlen(s)+1;
    data = new char[len]; // more memory
    strcpy(data, old);
    strcat(data, s);
    delete[] old;    // free old memory
}
\end{lstlisting}

\end{frame}



\begin{frame}[fragile]
    \begin{itemize}
        \item Automatic destruction at end of lifetime
        \item Destructors of members and base classes automatically called
            $ \implies $ simple composition of classes
        \item Immediate destructor call
        \item Allows other resources than memory:
        \begin{lstlisting}[language=c++, frame=single]
{
    lock_guard<mutex> guard(some_mutex);
    // ... code inside the mutex
} // some_mutex automatically unlocked
        \end{lstlisting}
    \end{itemize}
\end{frame}



\begin{frame}{Destruction Order}
    \begin{itemize}
        \item Destroy local variables at end of scope
        \item First run code of the destructor
        \\then destroy members and base classes
        \item Reverse of construction order
        \item
        \emph{Stack unwind} on raised exception: \\ Go back the call stack and destroy all objects.
    \end{itemize}
    
\end{frame}


\begin{frame}
    Already solved problems:
    \begin{itemize}
        \item Memory leaks
        \item double delete
        \item repeated resource release on all paths
        \item separation of allocation and release
        \item exceptions only with finally
    \end{itemize}
    
    Remaining and new problems:
    \begin{itemize}
        \item Use after free
        \item Strict hierarchies required; no cyclic references
    \end{itemize}
\end{frame}



\begin{frame}
    \frametitle{Containers}
    \begin{itemize}
        \item can store multiple objects of the same type
        \item use RAII to call all destructors
        \item can move objects internally
        \\ $\implies$ can cause dangling references / iterators
    \end{itemize} 
\end{frame}

\section{Smart Pointers}

\begin{frame}{Smart Pointers}
    \begin{definition}[Owning pointer]
        Pointer that may need to free its memory.
    \end{definition}
    \begin{definition}[Raw pointer]
        Pointer like in C, for example \emph{int*}.
    \end{definition}
    \begin{definition}[Smart pointer]
        Pointer-like object that manages its own memory.
    \end{definition}
\end{frame}

\begin{frame}[fragile]
    \frametitle{Unique Pointer}
    C++: \verb|unique_ptr| 
    
    Rust: \verb|Box|
    
    
    \begin{itemize}
        \item unique ownership
        \item hold a reference
        \item automatically deallocate it
    \end{itemize}
\end{frame}

\begin{frame}[fragile]
    \frametitle{Shared Pointer}
    C++: \verb|shared_ptr| 
    
    Rust: \verb|Rc|
  
    \begin{itemize}
        \item shared ownership
        \item uses reference counting
        \item increases counter in copy operation
        \item decreases counter in destructor and maybe destroys object
        \item reference cycles can cause leaks
    \end{itemize}
\end{frame}

\begin{frame}[fragile]
    \frametitle{Weak Pointer}
    C++: \verb|weak_ptr| 
    
    Rust: \verb|Weak|
    
    \begin{itemize}
        \item no ownership, pointer can dangle
        \item can be upgraded to a shared pointer
        \item used to break reference cycles
    \end{itemize}
\end{frame}


\begin{frame}
    \frametitle{without weak pointer}
    \begin{tikzpicture}[baseline=15]
    \tikz[scale=2] \graph [layered layout] {
        Root -> [dotted]
        Object 1 [draw, rectangle] 
        ->[bend right] Object 2 [draw, rectangle];
        
        Object 2 -> [bend right] Object 1;
        
        {[same layer] Object 1, Object 2};
    };
    \end{tikzpicture}
\end{frame}


\begin{frame}
    \frametitle{with weak pointer}
    \begin{tikzpicture}[baseline=15]
    \tikz[scale=2] \graph [layered layout] {
        Root -> [dotted]
        Object 1 [draw, rectangle]
        -> [bend right] Object 2 [draw, rectangle];
        
        Object 2 -> [bend right, dashed] Object 1;
        
        {[same layer] Object 1, Object 2};
    };
    \end{tikzpicture}
\end{frame}

\begin{frame}
    \frametitle{RAII in garbage collected languages}
    \begin{itemize}
        \item traditionally gc'ed languages use finally
        \item finally is more verbose than RAII and can be forgotten easily
        \item D uses garbage collector for memory and RAII for other resources
        \item Some languages RAII-like similar behavior
    \end{itemize}
\end{frame}


\begin{frame}[fragile]
    \frametitle{RAII in garbage collected languages}
    For example Python:
    \begin{lstlisting}[language=python,frame=single]
with open("test.file") as f:
    content = f.read()
    \end{lstlisting}
\end{frame}

\section{Rust's Ownership}

\begin{frame}
    \frametitle{Rust}
    
    Rust aims at memory safety, abstractions without overhead and multithreading.
    
    \begin{itemize}
        \item detect problems at compile time
        \item no dangling references        
        \item thread safety
        \item no overhead
    \end{itemize}
    It uses ownership, borrowing and lifetimes.
\end{frame}


\begin{frame}
    \frametitle{Ownership}
    
    \begin{itemize}
        \item only a single variable can access a value
        \item ownership can be transferred to another variable, e.g. inside a function
        \item can not use a variable after move
        \item some exceptions when transferring and copying is equal
    \end{itemize}
\end{frame}


\begin{frame}[fragile]
    \frametitle{Ownership}
    \begin{lstlisting}[frame=single]
    let a = vec![1, 2, 3];
    let b = a;
    
    // does not work
    a.push(4); // append 4
    \end{lstlisting}
\end{frame}


\begin{frame}
    \frametitle{Borrowing}
    
    \begin{itemize}
        \item temporarily borrow instead of transferring ownership
        \item like references
        \item move borrowed value is forbidden (no dangling references)
        \item similar to read-write-lock
        \item just one reference can mutate the variable \\
            $ \implies $ no dangling iterators
    \end{itemize}
\end{frame}


\begin{frame}[fragile, shrink]
    \frametitle{Borrowing}

\begin{lstlisting}[frame=single]
fn foo() -> &i32 {
    let a = 12;
    return &a;
}
\end{lstlisting}

\begin{lstlisting}[frame=single]
let x = vec![1,2,3];
let reference = &mut x;
println!("{}", x);
\end{lstlisting}

Borrowing is not threadsafe but can be used for it:
\begin{lstlisting}[frame=single]
{
    let guard = mutex.lock().unwrap();
    modify_value(&mut guard);
}
\end{lstlisting}

\end{frame}




\begin{frame}
    \frametitle{Lifetimes}
    
    \begin{itemize}
        \item mechanism for implementing ownership and borrowing
        \item every type has a lifetime
        \item can not take a reference with a larger lifetimes
        \item object can not be moved while a reference exists
        \item often they can be automatically generated
    \end{itemize}
\end{frame}

\begin{frame}[fragile]
    \frametitle{Lifetimes}
\begin{lstlisting}[frame=single]
struct Foo {
    x: i32,
}

// the lifetime 'a is a generic parameter
// it returns a reference with the same 
// livetime as the input
fn first_x<'a>(first: &'a Foo, second: &Foo)
     -> &'a i32 {
    &first.x
}
\end{lstlisting}
\end{frame}


\begin{frame}
    \frametitle{Anonymous functions}
    \hfill \hyperlink{skippedLambda}{\beamergotobutton{skip}}
    
    \begin{itemize}
        \item lambda functions
        \item can access local variable
        \item use ownership model for checks
        \item simplest type borrows environment
        \\ $\implies$ can not be returned
        \item second type takes ownership
        \\ $\implies$ variables can not be used from outside
    \end{itemize}
\end{frame}

\begin{frame}
    \frametitle{Anonymous functions}
    Some can be called only once.
    
    \begin{itemize}
        \item ownership is transferred into the called functions
        \item destroyed at end of call
        \item function can consume data
        \item same concept available for methods
    \end{itemize}
\end{frame}


\begin{frame}
    \frametitle{Anonymous functions in C++}    
    \begin{itemize}
        \item no checks by the compiler
        \item individual capture per value
        \item allows more complex expressions to be captured (C++14)
        \item more flexibility, less safety
        \item no self consuming functions in C++
    \end{itemize}
\end{frame}

\begin{frame}[fragile, shrink]
    \frametitle{Anonymous functions}
\begin{lstlisting}[frame=single]
let f = move || {
    // ...
};
\end{lstlisting}

\begin{lstlisting}[frame=single]
fn do_something(self) {
// ...
} // self is destroyed
\end{lstlisting}

\begin{lstlisting}[language=c++, frame=single]
[ x,  // by value
  &r, // by reference
  p = make_unique<int>(0)
      // generalized capture
] (auto parameter1) {
    // the code in the function
}
\end{lstlisting}
\end{frame}

\begin{frame}[label=skippedLambda]
    \frametitle{Ownership in C++}    
    \begin{itemize}
        \item ownership was already used for reasoning about code
        \item just the checks are new
        \item At CppCon 2015 a tool for checks similar to Rusts' was announced.
    \end{itemize}
\end{frame}


\section{Conclusion}


\begin{frame}
    \frametitle{Conclusion}
    RAII:
    \begin{itemize}
        \item safe building blocks for resource handling
        \item good program structure
        \item dangling references
        \item manually break cyclic references (not always possible)
        \item need to choose adequate type
    \end{itemize}
    RAII is general solution for resources without cycles \\ $\iff$ \\GC is general solution for memory handling
\end{frame}

\begin{frame}
    \frametitle{Conclusion}
    
    Ownership:
    \begin{itemize}
        \item prevents dangling references
        \item improves safety
        \item does not solve all issues
        \item requires lots of annotations
    \end{itemize}
\end{frame}

\begin{frame}        
        \Huge{Thank you for listening. \\
            Questions?}
\end{frame}

\end{document}